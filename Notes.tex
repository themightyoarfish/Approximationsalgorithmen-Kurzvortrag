\documentclass{article}
\usepackage[utf8x]{inputenc} 
\usepackage[ngerman]{babel} 
\usepackage{amsmath} 
\usepackage{caption} 

\title{FPTAS für das \emph{Restricted Shortest Path}-Problem}
\author{Rasmus Diederichsen \and Sebastian Höffner}

\begin{document}
\maketitle
\section{Problemstellung}

Das \emph{Shortest Path}-Problem besteht darin, in einem gewichteten Graphen den
kürzesten Weg von einem Start- zum Zielknoten auszurechnen. Wir kennen schon das
\emph{Single Source Shortest Path}-Problem, das darin besteht vom Startknoten
die kürzesten Wege zu allen anderen Knoten auszurechnen. Das Problem lässt sich
mit Dijkstras Algorithmus lösen. Weiterhin gibt es das \emph{All Pairs Shortest
Path}-Problem, für das wir Floyds Algorithmus kennengelernt haben. Hierbei
werden alle kürzesten Wege zwischen je zwei Knoten berechnet.

Unser Problem wandelt die obigen insofern ab, als Kanten nicht mehr nur Gewicht
$c_{ij}$ haben, sonden nun noch eine Verzögerung (\emph{delay} oder
\emph{transit time}) $t_{ij}$ zusätzlich haben.

Wir möchten den nach Gewichten kürzesten Pfad von Knoten $1$ nach $n$ berechnen,
dessen Verzögerung $\le T$ ist.

\section{Exakte Lösung}

Durch dynamische Programmierung kann das Problem gelöst werden.
Sei $g_j(c)$ die Gesamtverzögerund des schnellsten Pfades von $1$ zu $j$, der
nicht länger als $c$ ist.

\subsection{Algorithmus B}

\begin{align*}
   g_1(c) & =  0,                                                                                            & c = 0,\ldots,OPT, \\
   g_j(0) & =  \infty,                                                                                       & j = 2,\ldots,n, \\
   g_j(c) & =  \min\left\{g_j(c-1), \min_{k \mid c_{kj} \le
   c}\left\{g_k(c-c_{kj}) + t_{kj}\right\}\right\}, & j = 2,\ldots,n; ~ c= 1,\ldots,OPT
\end{align*}
\captionof{figure}{Algorithmus B}

\vskip\baselineskip

Die Laufzeit ist hier $\mathcal{O}(OPT\cdot n \cdot \text{Aufwand pro
$(c,j)$})$. Für jedes $g_j(c)$ kann ein Aufwand von $\mathcal{O}(n)$ anfallen,
da alle Vorgängerknoten durchsucht werden müssen. Es ergibt sich eine Laufzeit
von $\mathcal{O}(n^2 OPT) = \mathcal{O}(|E| OPT)$.

\subsection{Wann terminiert der Algorithmus?}

Man kennt zwar $OPT$ nicht, aber man weiß, dass $OPT=\min\left\{c \mid g_n(c)
\le T\right\}$, also setzt man $OPT$, soblad man das erste $c$ gefunden hat mit
$g_n(c)\le T$, denn schneller kann es per Definition von $g$ nicht werden.

Das Problem ist, dass der Algorithmus nur pseudopolynomiell ist (vgl. Rucksack).
die Kantengewichte und damit $OPT$ können aber exponentiell in der Eingabe(bit)länge sein.

\end{document}
