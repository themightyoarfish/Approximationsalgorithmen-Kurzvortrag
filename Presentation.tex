\documentclass{beamer}
\usepackage[utf8x]{inputenc}
\usepackage[ngerman]{babel} 
\usepackage{amsmath} 
\usepackage{caption} 
\usepackage{xcolor} 
\usepackage{listings}
\usepackage{wasysym}
\lstset{
   xleftmargin  = 3em,
   escapeinside = {*}{*},
   gobble=6,
   basicstyle   = \footnotesize\ttfamily,
   breaklines   = true,
   commentstyle = \color{blue},
   keywordstyle = \color{purple}\textbf,
   numberstyle  = \small\color{gray},
   numbers      = left,
   stringstyle  = \color{olive},
}
\usepackage{array}
\usepackage{tikz}
\newcommand{\trianglearrow}{{\tiny\usebeamercolor[bg]{block title} \raisebox{.5ex}{$\blacktriangleright$} }} % hacky

\usetheme{UOS}
\useinnertheme{circles}
\useoutertheme{split}
\setbeamertemplate{itemize subitem}[triangle]

\title{FPTAS für das Restricted Shortest Path-Problem nach (Hassin 1992)}
\author{Rasmus Diederichsen \and Sebastian Höffner}
\institute{Universität Osnabrück}
\date{\today}

\begin{document}
\frame{\titlepage}
\begin{frame}
   \frametitle{Inhalt}
   \tableofcontents
\end{frame}

\section{Das Problem}
\begin{frame}
   \frametitle{Problemstellung}
   \begin{block}{Gegeben}
      \begin{itemize}
         \item azyklischer, gerichteter Graph $G = (V,E)$
         \item $(u,v) \in E$ hat Gewicht $c$ und Verzögerung $t$
      \end{itemize}
   \end{block}
   \only<1>
   {
      \begin{block}{Single Source Shortest Path}
         Berechne vom Startknoten aus alle nach Kosten kürzesten Wege zu allen
         anderen \trianglearrow Dijkstra
      \end{block}
      \begin{block}{All Pairs Shortest Path}
         Kürzeste Wege zwischen allen Knotenpaaren \trianglearrow Floyd
      \end{block}
   }
   \only<2>
   {
      \begin{alertblock}{Restricted Shortest Path}
         Finde nach Kosten kürzesten Weg von $a$ nach $b$ mit Verzögerung $\le T$.
         \textbf{NP}-schwer.
      \end{alertblock}
   }
\end{frame}

% \begin{frame}
%    \frametitle{\insertsection}
%    \framesubtitle{\insertsubsection}
%    \begin{block}{Gegeben}
%       \begin{itemize}
%          \item azyklischer gerichteter Graph $G = (V,E)$
%          \item $(u,v) \in E$ hat Gewicht $c$ und Verzögerung $t$
%       \end{itemize}
%    \end{block}
%    \begin{alertblock}{Restricted Shortest Path}
%       Finde nach Kosten kürzesten Weg von $a$ nach $b$ mit Verzögerung $\le T$.
%       \textbf{NP}-schwer.
%    \end{alertblock}
% \end{frame}

\section{Exakte Lösung}
\subsection{Algorithmus}
\begin{frame}
   \frametitle{\insertsection}
   \framesubtitle{\insertsubsection}

   Dynamische Programmierung (ähnlich wie Knapsack). Kanten $(i,j)$ mit $i < j$,
   da azyklisch.
   \begin{block}{Algorithmus}
      \begin{align*}
         g_1(c) & =  0, ~ \text{Für} ~ c = 0,\ldots,OPT, \\
         g_j(0) & =  \infty, ~ \text{Für} ~ j = 2,\ldots,n, \\
         g_j(c) & =  \min\left\{g_j(c-1), \min_{k \mid c_{kj} \le
   c}\left\{g_k(c-c_{kj}) + t_{kj}\right\}\right\} \\
                & \text{Für} ~ c= 1,\ldots,OPT; ~ j = 2,\ldots,n
\end{align*}
   \end{block}
\end{frame}

\subsection{Laufzeit}

\begin{frame}
   \frametitle{\insertsection}
   \framesubtitle{\insertsubsection}
   \begin{beamercolorbox}[center,wd=\paperwidth]{block body}
      \begin{align*}
         g_1(c) & =  0, ~ \text{Für} ~ c = 0,\ldots,OPT, \\
         g_j(0) & =  \infty, ~ \text{Für} ~ j = 2,\ldots,n, \\
         g_j(c) & =  \min\left\{g_j(c-1), \min_{k \mid c_{kj} \le c}\left\{g_k(c-c_{kj}) + t_{kj}\right\}\right\} \\
                & \text{Für} ~ c= 1,\ldots,OPT; ~ j = 2,\ldots,n
      \end{align*}
   \end{beamercolorbox}

   \begin{itemize}
      \item $\mathcal{O}(OPT\cdot n \cdot \text{Aufwand pro $(c,j)$})$
         \begin{itemize}
            \item Pro $(c,j)$ alle direkten Vorgänger betrachten, alle Kanten
               max. $2$-mal
            % \item $\mathcal{O}\left(n^2 OPT\right) = \mathcal{O}\left(|E| OPT\right)$
            \item $\mathcal{O}\left(2|E| OPT\right) = \mathcal{O}\left(|E| OPT\right)$
         \end{itemize}
      \item Pseudopolynomiell
   \end{itemize}

\end{frame}

\subsection{Terminierung}

\begin{frame}
   \frametitle{\insertsection}
   \framesubtitle{\insertsubsection}

   \begin{beamercolorbox}[center,wd=\paperwidth]{block body}
      \begin{align*}
         g_1(c) & =  0, ~ \text{Für} ~ c = 0,\ldots,OPT, \\
         g_j(0) & =  \infty, ~ \text{Für} ~ j = 2,\ldots,n, \\
         g_j(c) & =  \min\left\{g_j(c-1), \min_{k \mid c_{kj} \le c}\left\{g_k(c-c_{kj}) + t_{kj}\right\}\right\} \\
                & \text{Für} ~ c= 1,\ldots,OPT; ~j = 2,\ldots,n
      \end{align*}
   \end{beamercolorbox}

   Man weiß $OPT=\min\left\{c \mid g_n(c) \le T\right\}$
   \begin{itemize}
      \item Setze $OPT$, sobald erstes $c$ mit $g_n(c)\le T$ gefunden.
   \end{itemize}
\end{frame}


\subsection{Beispiel}

\begin{frame}
   \frametitle{\insertsection}
   \framesubtitle{\insertsubsection}
   \setbeamercolor{bgcolor}{fg=black,bg=\bgcolor}
   \begin{block}{Autobahn? Oder doch lieber Landstraße?}
      \begin{center}
         \tikzstyle{vertex}=[circle,thick,draw=block title.bg]
         \begin{tikzpicture}[node distance=2.5cm,auto,
            vertex/.append style={fill=white}]
            \node[vertex] (1) {1};
            \node[vertex] (2) [right of=1] {2};
            \node[vertex] (3) [below right of=2] {3};
            \node[vertex] (4) [above right of=2] {4};
            \node[vertex] (5) [below right of=4] {5};
            \node[vertex] (6) [right of=5] {6};
            \draw [-to] (1) -- 
            node [above] {$t_{12}=1$} 
            node [below] {$c_{12}=1$}
            (2);
            \draw [-to] (2) -- 
            node [above] {$2$} 
            node [below] {$1$}
            (3);
            \draw [-to] (2) -- 
            node [above] {$1$} 
            node [below] {$2$}
            (4);
            \draw [-to] (3) -- 
            node [above] {$1$} 
            node [below] {$1$}
            (5);
            \draw [-to] (4) -- 
            node [above] {$1$} 
            node [below] {$1$}
            (5);
            \draw [-to] (5) -- 
            node [above] {$1$} 
            node [below] {$1$}
            (6);
         \end{tikzpicture}
      \end{center}
   \end{block}
\end{frame}

\begin{frame}
   \frametitle{\insertsection}
   \framesubtitle{\insertsubsection}
   \begin{center}
      \begin{tabular}{>{$}r<{$}|>{$}r<{$}>{$}r<{$}>{$}r<{$}>{$}r<{$}>{$}r<{$}>{$}r<{$}}
         j\backslash c & 0 & 1 & 2 & 3 & 4 & 5  \\
         \hline
         1 & 0 & 0 & 0 & 0 & 0 & 0 \\
         2 & \infty & \infty & \infty & \infty & \infty & \infty \\
         3 & \infty & \infty & \infty & \infty & \infty & \infty \\
         4 & \infty & \infty & \infty & \infty & \infty & \infty \\
         5 & \infty & \infty & \infty & \infty & \infty & \infty \\
         6 & \infty & \infty & \infty & \infty & \infty & \infty 
      \end{tabular}
   \end{center}
\end{frame}

\begin{frame}
   \frametitle{\insertsection}
   \framesubtitle{\insertsubsection}
   \begin{center}
      \begin{tabular}{>{$}r<{$}|>{$}r<{$}>{$}r<{$}>{$}r<{$}>{$}r<{$}>{$}r<{$}>{$}r<{$}}
         j\backslash c & 0 & 1 & 2 & 3 & 4 & 5  \\
         \hline
         1 & 0 & 0 & 0 & 0 & 0 & 0 \\
         2 & \infty & 1 & \infty & \infty & \infty & \infty \\
         3 & \infty & \infty & \infty & \infty & \infty & \infty \\
         4 & \infty & \infty & \infty & \infty & \infty & \infty \\
         5 & \infty & \infty & \infty & \infty & \infty & \infty \\
         6 & \infty & \infty & \infty & \infty & \infty & \infty 
      \end{tabular}
   \end{center}
   \begin{align*}
      g_2(1) &= \min\left\{ g_2(0), 
      \min\limits_{k|c_{kj}\leq c}\left\{
         g_k\left(c-c_{kj}\right) + t_{kj}
      \right\}
   \right\} \\
   g_2(1) &= \min\left\{ \infty, 
   \min\left\{
      g_1\left(1-1\right) + 1
   \right\}
\right\} \\
g_2(1) &= 1
   \end{align*}
\end{frame}


\begin{frame}
   \frametitle{\insertsection}
   \framesubtitle{\insertsubsection}
   \begin{center}
      \begin{tabular}{>{$}r<{$}|>{$}r<{$}>{$}r<{$}>{$}r<{$}>{$}r<{$}>{$}r<{$}>{$}r<{$}}
         j\backslash c & 0 & 1 & 2 & 3 & 4 & 5  \\
         \hline
         1 & 0 & 0 & 0 & 0 & 0 & 0 \\
         2 & \infty & 1 & 1 & 1 & 1 & 1 \\
         3 & \infty & \infty & 3 & 3 & 3 & 3 \\
         4 & \infty & \infty & \infty & 2 & 2 & 2 \\
         5 & \infty & \infty & \infty & 4 & 3 & 3 \\
         6 & \infty & \infty & \infty & \infty & 5 & \textbf{4} 
      \end{tabular}
   \end{center}
\end{frame}

\section{Das FPTAS}

\subsection{Test für Grenzen von $OPT$}

\begin{frame}
   \frametitle{\insertsection}
   \framesubtitle{\insertsubsection}

   Wir suchen zunächst ein Verfahren, dass untere und obere Schranken für $OPT$
   findet.
   \begin{itemize}
      \item Wünsch-dir-was: Polynomieller Algorithmus $TEST(k)$, sodass
         \begin{align*}
            TEST_{magic}(k) = 
            \begin{cases}
               1 & \text{falls } OPT \ge k \\
               0 & \text{falls } OPT < k \\
            \end{cases} 
         \end{align*}
         \begin{itemize}
            \item Binäre Suche auf ${0,\ldots,UB}$ 
            \item Leider \textbf{NP}-schwer
         \end{itemize}
   \end{itemize}
\end{frame}

\begin{frame}
   \frametitle{\insertsection}
   \framesubtitle{\insertsubsection}

   $TEST_{magic}(k)$ kann nicht existieren, also schwächer:
   \only<1>{
      \begin{block}{Eigenschaften von $TEST(k)$}
         \begin{align*}
            TEST(k) = 
            \begin{cases}
               1 & \text{falls } OPT \ge k \\
               0 & \text{falls } OPT < k(1 + \epsilon) \\
            \end{cases} 
         \end{align*}
      \end{block}
   }

   \only<2>{
      \begin{block}{$TEST(k)$}
         \begin{itemize}
            \item Skaliere und runde Kantengewichte als $\hat{c}_{ij} = \left\lfloor\frac{c_{ij} (n-1)}{k\epsilon}\right\rfloor$
            \item Wende exakten Algorithmus an, bis $g_n(c) \le T$ gefunden ist 
               %für $c < \frac{n-1}{\epsilon}$ 
               oder $c \ge \frac{n-1}{\epsilon}$.
         \end{itemize}
      \end{block}
   }

\end{frame}

\begin{frame}
   \frametitle{\insertsection}
   \framesubtitle{\insertsubsection}

   $TEST(k)$ erfüllt seinen Zweck:
   \only<1>
   {
      \begin{block}{$c<\frac{n-1}{\epsilon}$ $\rightarrow$ Es gibt Pfad $ < k(1+\epsilon)$}
         \small
         \setlength\abovedisplayskip{-12pt}
         \begin{align*}
            k &\leq k \\
            \frac{k\epsilon}{n-1} \frac{n-1}{\epsilon} &\leq k
            \intertext{Durch Einsetzen folgt:}
            \frac{k\epsilon}{n-1} c & < k \\
            \frac{k\epsilon}{n-1} c + k\epsilon & < k + k\epsilon\\
            \frac{k\epsilon}{n-1} c + k\epsilon & < k(1+\epsilon)
         \end{align*}
      \end{block}
   }
   \only<2>
   {
      \begin{block}{$c\geq\frac{n-1}{\epsilon}$ $\rightarrow$ Jeder $T$-Pfad hat
         Länge $\ge k$}
         \begin{align*}
            c & \ge \frac{n-1}{\epsilon} \\
            c \frac{k\epsilon}{n-1} & \ge \frac{k\epsilon}{n-1}\frac{n-1}{\epsilon} \\
            c \frac{k\epsilon}{n-1} & \ge k
         \end{align*}
      \end{block}
   }
\end{frame}

\begin{frame}[fragile]
   \frametitle{\insertsection}
   \framesubtitle{\insertsubsection}
   \begin{block}{Test-Algorithmus}
      \begin{lstlisting}
      Setze *$c \leftarrow 0$*
      F*ü*r alle *$(i,j)\in E$*:
         Falls *$c_{ij}$ > k*, entferne *$(i,j)$*
         Sonst *$c_{ij} \leftarrow \lfloor c_{ij}(n-1) / k\epsilon\rfloor$*

      Falls *$c \ge (n-1) / \epsilon$*, return true
      Sonst: 
         Wende Algorithmus B an und berechne *$g_j(c)$* f*ü*r *$j=2,\ldots,n$*
         Falls *$g_n(c) \le T$*, return false
         Sonst:
            Setze *$c \leftarrow c + 1$*
            Goto Zeile 6
      \end{lstlisting}
   \end{block}
\end{frame}

\subsection{Laufzeit des Tests}

\begin{frame}
   \frametitle{\insertsection}
   \framesubtitle{\insertsubsection}
   \begin{itemize}
      \item Runden: in $\mathcal{O}(\log{n})$ durch binäre Suche, falls nach
         oben beschränkt \trianglearrow $\mathcal{O}\left(|E|\log{\frac{n-1}{\epsilon}}\right)$
      \item Exakter Algorithmus führt $\le \frac{n-1}{\epsilon}$ Iterationen
         durch, $\mathcal{O}\left(|E|\right)$ pro Iteration
         \begin{itemize}
            \item Insgesamt $\mathcal{O}\left(|E|\log{\frac{n-1}{\epsilon}} + |E|\frac{n-1}{\epsilon}\right) = \mathcal{O}\left(|E|\frac{n-1}{\epsilon}\right)$
         \end{itemize}
         \item Runden und wieder Multiplizieren verringern Kantenkosten um
            höchstens $\frac{k\epsilon}{n-1}$, Kosten über $n-1$ Kanten also
            max. $k\epsilon$.
   \end{itemize}

\end{frame}

\subsection{Verbesserte Grenzen von $OPT$}

\begin{frame}
   \frametitle{\insertsection}
   \framesubtitle{\insertsubsection}
   Wir benötigen obere und untere Schranken $LB \le OPT \le UB$.
   \only<1->{
      \begin{block}{Lower Bound}
         \begin{itemize}
            \item $LB = 1$
            \item $LB =$ kürzester Pfad nach Kosten
         \end{itemize}
      \end{block}
   }
  
   \only<2>{
      \begin{block}{Upper Bound}
         \begin{itemize}
            \item $UB = \sum (n-1)~ \text{längste Kanten}$
            \item $UB =$ Kosten schnellster Pfad von $1$ nach $n$.
         \end{itemize}
      \end{block}
   }
\end{frame}

\subsection{Algorithmus}

\begin{frame}[fragile]
   \frametitle{\insertsection}
   \framesubtitle{\insertsubsection}
   \begin{block}{Approximationsschema-Algorithmus}
      \begin{lstlisting}
      *$UB := \sum (n-1) \text{ größte Kosten}$*
      *$LB := 1$*

      Falls *$UB \le 2LB$*, Goto Zeile 11
      Sonst:
         *$k := \sqrt{UB \cdot LB}$*
         Falls *$TEST(k)$* == true, *$LB := k$*
         Sonst *$UB := k(1 + \epsilon)$*
         Goto Zeile 4

      Setze *$c_{ij} \leftarrow c_{ij} (n-1)/LB\epsilon$*
      Berechne optimale L*ö*sung
      \end{lstlisting}
   \end{block}
\end{frame}

\subsection{Laufzeit des FPTAS}
\begin{frame}
   \frametitle{\insertsection}
   \framesubtitle{\insertsubsection}
   \begin{itemize}[<+->]
      \item Wie für $TEST$ kann Abweichung vom $OPT$ nicht größer
         als $k\epsilon$ sein, hier mit $k=LB$, die Abweichung ist also $\le
         OPT\epsilon$
      \item Rundung der Kantenkosten: Laufzeit letzte Anwendung
         des exakten Algorithmus' ist $\mathcal{O}\left(|E| OPT\frac{(n-1)}{LB\epsilon}\right)$
      \item $OPT \le 2LB$: $\mathcal{O}(|E| 2LB\frac{(n-1)}{LB\epsilon})=\mathcal{O}(|E|
         2\frac{(n-1)}{\epsilon})=\mathcal{O}\left(|E|\frac{n-1}{\epsilon}\right)$
      \item Laut Hassin $\log{\log{\frac{UB}{LB}}}$ Tests, bis $\frac{UB}{LB} \le 2$
         \earth
      \item Wurzeln evtl. teuer, es reicht aber $\log{\log{\frac{UB}{LB}}}$ \earth
      \item Einzelne Aufrufe von $TEST$: $\mathcal{O}\left(|E|\frac{n-1}{\epsilon}\right)$
      \item Insgesamt $\mathcal{O}(\log\log{\frac{UB}{LB}\cdot(|E|\frac{n-1}{\epsilon} +
         \log\log{\frac{UB}{LB}})})$
   \end{itemize}
\end{frame}

\begin{frame}
   \centerline{Fin.}
\end{frame}
\end{document}
